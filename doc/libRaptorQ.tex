%    Copyright (C)  2015, Luca Fulchir<luca@fulchir.it>
%    Permission is granted to copy, distribute and/or modify this document
%    under the terms of the GNU Free Documentation License, Version 1.3
%    or any later version published by the Free Software Foundation;
%    with no Invariant Sections, no Front-Cover Texts, and no Back-Cover Texts.
%    A copy of the license is included in the section entitled "GNU
%    Free Documentation License".



\documentclass[11pt,a4paper]{refart}
\usepackage[utf8]{inputenc}
\usepackage{hyperref}
\usepackage{makeidx}
\usepackage{verbatimbox}
\usepackage{listings}
\usepackage{color}

\definecolor{mygreen}{rgb}{0,0.6,0}
\definecolor{mygray}{rgb}{0.5,0.5,0.5}
\definecolor{mymauve}{rgb}{0.58,0,0.82}

\lstset{ %
  backgroundcolor=\color{white},   % choose the background color; you must add \usepackage{color} or \usepackage{xcolor}
  basicstyle=\footnotesize,        % the size of the fonts that are used for the code
  breakatwhitespace=false,         % sets if automatic breaks should only happen at whitespace
  breaklines=true,                 % sets automatic line breaking
  captionpos=b,                    % sets the caption-position to bottom
  commentstyle=\color{mygreen},    % comment style
  deletekeywords={...},            % if you want to delete keywords from the given language
  escapeinside={\%*}{*)},          % if you want to add LaTeX within your code
  extendedchars=true,              % lets you use non-ASCII characters; for 8-bits encodings only, does not work with UTF-8
  frame=single,                    % adds a frame around the code
  keepspaces=true,                 % keeps spaces in text, useful for keeping indentation of code (possibly needs columns=flexible)
  keywordstyle=\color{blue},       % keyword style
  language=C,                 % the language of the code
  %otherkeywords={*,...},            % if you want to add more keywords to the set
  %numbers=left,                    % where to put the line-numbers; possible values are (none, left, right)
  %numbersep=5pt,                   % how far the line-numbers are from the code
  %numberstyle=\tiny\color{mygray}, % the style that is used for the line-numbers
  rulecolor=\color{black},         % if not set, the frame-color may be changed on line-breaks within not-black text (e.g. comments (green here))
  showspaces=false,                % show spaces everywhere adding particular underscores; it overrides 'showstringspaces'
  showstringspaces=false,          % underline spaces within strings only
  showtabs=false,                  % show tabs within strings adding particular underscores
  stepnumber=2,                    % the step between two line-numbers. If it's 1, each line will be numbered
  stringstyle=\color{mymauve},     % string literal style
  tabsize=4,                       % sets default tabsize to 2 spaces
  title=\lstname                   % show the filename of files included with \lstinputlisting; also try caption instead of title
}

\title{Using libRaptorQ library}
\author{Luca Fulchir -- \{luker\}@fenrirproject.org}
\date{\today}
\makeindex

\begin{document}
\maketitle

\begin{abstract}

\textbf{libRaptorQ} is a C++11 implementation of the RaptorQ Forward Error Correction, as described in the \href{https://tools.ietf.org/html/rfc6330}{RFC6330} .

The implementation was started as a university laboratory project, and will be later used and included in \href{https://www.fenrirproject.org}{Fenrir}, the
maintainer's master thesis.

This implementation is quite short (the core is $\sim3k$ lines), thanks to the chosen language and the use of external libraries for matrix handling (eigen3).

libRaptorQ is the only RaptorQ implementation in C++, include C hooks, and it is the only free (\textbf{LGPL3}) implementation of the rfc, except for the (apache2)
java implementation, OpenRQ , which is much bigger ($\sim 46k$) and slower.
\end{abstract}

\vfill\hfill
\begin{verbbox}[\tiny]
Copyright (C)  2015, Luca Fulchir<luca@fulchir.it>
Permission is granted to copy, distribute and/or modify this document
under the terms of the GNU Free Documentation License, Version 1.3
or any later version published by the Free Software Foundation;
with no Invariant Sections, no Front-Cover Texts, and no Back-Cover Texts.
A copy of the license is included in the section entitled "GNU
Free Documentation License".
\end{verbbox}
\theverbbox

\newpage
\tableofcontents
\newpage

%%%%%%%%%%%%%%%%%%%%%%%%%%%%%%%%%%%%%%%%%%%%%%%%%%%%%%%%%%%%%%%%%%%%%%%%%%%
%%%%%%%%%%%%%%%%%%%%%%%%%%%%%%%%%%%%%%%%%%%%%%%%%%%%%%%%%%%%%%%%%%%%%%%%%%%
\section{Build \& install}
\subsection{Get the source code}
Although things seems to work, no stable release has been released yet, as of \today.

This means you can only check this out with git.

To check out the repository:

\begin{verbatim}
$ git clone https://github.com/LucaFulchir/libRaptorQ.git
\end{verbatim}

You can also get it from our main server:

\begin{verbatim}
$ git clone git://git.fenrirproject.org/libRaptorQ.git
\end{verbatim}

\index{GPG}\marginlabel{GPG verification:}
Once you have cloned it, it's always a good thing to check the repository gpg
signatures, so you can import my key with:

\begin{verbatim}
$ gpg --keyserver pgp.mit.edu --recv-key D42DDF0A
\end{verbatim}

please check the full fingerprint, it should be this:

\begin{verbbox}[\footnotesize]
$ gpg2 --fingerprint D42DDF0A
 pub   rsa3072/D42DDF0A 2015-01-01 [expires: 2016-01-01]
       Key fingerprint = AB35 E45F 5CA5 E35B 8B55  818F 0157 D133 D42D DF0A
 uid       [ unknown] Luca Fulchir (2015 key) <luker@fenrirproject.org>
\end{verbbox}
\theverbbox
 

Now you have the source, and the key, it's enough to check the signature of the
last commit:

\begin{verbatim}
$ git log -n 1 --show-signature
\end{verbatim}

The important part is that you get something like this:

\begin{verbbox}[\footnotesize]
 gpg: Signature made Fri 27 Mar 2015 20:59:59 CET using RSA key ID D42DDF0A
 gpg: Good signature from "Luca Fulchir (2015 key) <luker@fenrirproject.org>"
 [unknown]
 gpg: WARNING: This key is not certified with a trusted signature!
 gpg:          There is no indication that the signature belongs to the owner.
 Primary key fingerprint: AB35 E45F 5CA5 E35B 8B55  818F 0157 D133 D42D DF0A
 Author: Luca Fulchir <luker@fenrirproject.org>
\end{verbbox}
\theverbbox

And as long as you got the right key, and you find the \textbf{"gpg: Good signature"} string,
you can be sure you have the right code.

\subsection{Dependencies}\index{dependencies}

libRaptorQ has only 2 dependencies:
\begin{description}
\item[\textbf{Eigen}] This is used for matrix manipulation, which is a big part of RaptorQ.
\item[\textbf{git}] This is used not only to get the source, but also by the build system. We get the last git commit id and feed it to clang or gcc as seed for their
internal random number generator. This makes it possible to have reproducible builds.
\end{description}

\subsection{Build \& Install}

libRaptorQ uses the cMake build system, so things are fairly standard:

\begin{verbatim}
$ cd libRaptorQ.git
$ mkdir build
$ cmake ../
$ make -j 4
\end{verbatim}

\index{cMake}
There are lots of options, you can use in cmake. As always, you can change them by adding ``\textbf{-Dcmake\_option=cmake\_value}'' when calling cmake.

The ones we recognize are:

\begin{description}
\item[CMAKE\_CXX\_COMPILER] g++, clang++ are directly supported. other should work, too.
\item[STDLIB] to change the c++ standard library. ``libstdc++'' for the (default) gcc one, ``libc++'' for the clang/llvm one. Note that it seems you can't use libc++ with gcc as of yet.
\item[CMAKE\_CXX\_FLAGS] Additional compiler flags you might want to pass.
\item[CMAKE\_BUILD\_TYPE] Type of build. you can choose between ``Debug'', ``Release'' or ``MinSizeRel''
\item[CMAKE\_INSTALL\_PREFIX] Default to \textit{/usr/local}. Change it to fit your distribution guidelines.
\end{description}

Then you can build everything by running:
\begin{verbatim}
$ make -j 4
\end{verbatim}
Of course, you can configure the number of parallel jobs (the \textit{-j} parameter) to be what you need.

If you want to build this documentation, too, you need to have installed latex. Then run:
\begin{verbatim}
$ make docs
\end{verbatim}

~\\

\index{Install}\marginlabel{\textbf{Install:}}
The installation process is very simple:

\begin{verbatim}
$ make install DESTDIR=/usr
\end{verbatim}

You can change the \textit{DESTDIR} parameter to fit your distribution guidelines.
\newpage

%%%%%%%%%%%%%%%%%%%%%%%%%%%%%%%%%%%%%%%%%%%%%%%%%%%%%%%%%%%%%%%%%%%%%%%%%%%
%%%%%%%%%%%%%%%%%%%%%%%%%%%%%%%%%%%%%%%%%%%%%%%%%%%%%%%%%%%%%%%%%%%%%%%%%%%

\section{Working with RaptorQ}
\subsection{Theory (you really need this)}\index{Theory}

To be able to work with liRaptorQ, you must first understand how the RaptorQ algorithms works. We won't go into the details, but just what you need
to be able to tune the algorithm to your needs.


\marginlabel{Fountain codes:}
Fountain codes are a special \textit{Forward-Error-Correcting} code class, which characteristic is simple: if you want to send $K$ packets, you actually
send $K+X$ packets, and the receiver only needs to get any $K$ packets to be able to reconstruct your data. The number $X$ of overhead packets can be as big
as you need (theoretically infinite), so you can tune it to be slightly higher than the expected packet loss.

\marginlabel{Systematic codes:} RaptorQ is also a systematic code. This means that those first $K$ packets are the input \textit{as-is} (\textbf{source symbols}),
and the $X$ packets (\textbf{repair symbols}) have the information needed to recover any of the lost source packets. This has the big advantage of
avoiding any kind of time and memory consuming decoding if there is no packet loss during the transmission.

\marginlabel{Complexity:} The RaptorQ algorithm is presented as having a linear time encoder and decoder. This implementation does not seem to verify
this property, and the algorithm itself for now is actually cubic on the number of symbols.\\
It is still very quick. On a core i7, 2.4Ghz, you need to wait \textit{0.4ms} for $10$ symbols, \textit{280ms} for 1.000 symbols, but it can take an hour for $27.000$ symbols.
RaptorQ handles up to $56.403$ symbols.

\subsection{Blocks \& Symbols}

To understand how to properly use and tune libRaptorQ, you first need to understand how RaptorQ handles its inputs, outputs, and what the time and memory
constraints are.

\index{Sequencing}\marginlabel{Input sequencing:}
RaptorQ needs to have the whole input you want to send before it can start working.\\
This means that it might be a little more difficult to use in live-streaming contexts, or where you need real-time data, but libRaptorQ will have options to
facilitate usage even in those contexts.

Once you have the whole input, RaptorQ divides it into \textbf{blocks}. Each block \textit{is encoded and decoded independently} and will be divided into \textbf{symbols}. Each symbol \textit{should}
be transmitted separately in its own packet (if you are on the network).

\index{Sequencing!Sizes}\marginlabel{Sizes:}Each input can have up to \textit{256 blocks}, each block can have up to \textit{$56.403$ symbols}, and each
symbol can have up to \textit{$2^{16}-1$ bytes}long. This gives a maximum files size of almost $881$ GB (946270874880 bytes to be exact)

\index{Interleaving}\marginlabel{Interleaving:}
An other feature of RaptorQ is to automatically provide some interleaving of the input data before transmitting it. This means that one symbol will not
represent one sequential chunk of your input data, but more likely it's the concatenation of different \textbf{sub-symbols}. The size of the subsymbol must thus
be a fraction of the symbol size. This feature is not used if you set the size of the subsymbol to the size of symbol.


\marginlabel{Memory and Time:}
Memory and time requirements are to be considered, though, as RaptorQ needs to run a cubic algorithm on matrix of size $K*K$, where $K$ is the number of
symbols in each block.\\
The algorithm needs to keep in memory two of these matrices, although most of the work is done on only one.\\
This is actually a lot. More benchmarks and optimizations will come later, for now remember that with 10 symbols it takes something like 0.4ms on a core i7 2.4GHZ, 280ms with 1000 symbols, and up to an hour with 27.000 symbols.


\subsection{C++ interface}
\index{Interface!C++}
To use the C++ interface you only need to include the \textbf{RaptorQ.hpp} header.

To provide grater flexibility, the whole library uses iterators to read your data, and to write the data onto your data structures.\\
This means that a big part of the library is a template, which adapts to the alignment of the data in the data structures you use.

\index{Iterators}\marginlabel{Templates}
There are two main classes you will use:
\begin{verbatim}
template <typename Rnd_It, typename Out_It>
class Encoder

template <typename In_It, typename Out_It>
class Decoder
\end{verbatim}

As you might guess, the classes for the encoder and decoder take two template parameters.\\
For the \textbf{Encoder}, the first parameter \textit{MUST} be a \textit{random access iterator} to the input data, and the second parameter is an
\textit{output iterator}. The random access iterator will be used to scan your input data, and perform an interleaving (if you did not set the same size the
symbol and to the subsymbol). The output iterator will be used to write the data to your structure.\\
The same is done for the \textbf{Decoder}, but we do not need to do any interleaving on the input, so the first iterator can be just an input iterator,
and nothing more.

~\\

\subsubsection{The Encoder}
\index{Encoder!C++}
You can instantiate a decoder for example by doing:

\begin{verbbox}[\small]
 std::vector<uint32_t> input, output;
 ...
 using T_it = typename std::vector<uint32_t>::iterator;
 RaptorQ::Encoder<T_it, T_it> enc (input.begin(), input.end(),
         4, 1444, 10000)
\end{verbbox}
\theverbbox

This will create an Encoder that works on vectors of unsigned 32 bit integers for both input and output, that will create symbols of size 1444 bytes, interleaving
your input every 4 bytes, and try to work with big number of symbols per blocks (\textbf{TODO: explain memory requirements})

The available methods for the encoder are the following:

\begin{description}
\item[bool operator()()] \textbf{return:bool}\\
False if constructor parameters did not make sense. Else true.

\item[OTI\_Common()] \textbf{return: OTI\_Common\_Data}, aka \textit{uint64\_t}.\\
Keeps total \textbf{file size and symbol size}. You need to send this to the receiver, so that it will be able to properly decode the data.

\item[OTI\_Scheme\_Specific\_Data()] \textbf{return: OTI\_Scheme\_Specific\_Data}, aka \textit{uint32\_t}.\\
Keeps number of \textbf{source blocks, sub blocks, and alignment}. As for the OTI\_Common\_Data, you need to send this to the receiver to be able to
properly decode the data.

\item[encode] \textbf{Input: Out\_It \&output, const Out\_It end, const uint32\_t esi, const uint8\_t sbn}.\\
\textbf{return:uint64\_t}.\\
Take as input the iterators to the data structure into where we have to save the encoded data, the \textbf{Encoding Symbol Id} and the
\textbf{Source Block Number}. As you are writing in C++, you probably want to use the iterators begin/end, though. Returns the number of written
iterators (\textbf{NOT} the bytes)

\item[encode] \textbf{Input: Out\_It \&output, const Out\_It end, const uint32\_t id}.\\
\textbf{return:uint64\_t}.\\
Exactly as before, but the \textbf{id} contains both the \textit{source block number} and the \textit{encoding symbol id}

\item[begin()] \textbf{return: Block\_Iterator<Rnd\_It, Out\_It>}\\
This returns an iterator to the blocks in which RaptorQ divided the input data. See later to understand how to use it.
\item[end()] \textbf{return: const Block\_Iterator<Rnd\_It, Out\_It>}\\
This returns an iterator to the end of the blocks in which RaptorQ divided the input data. See later to understand how to use it.

\item[precompute] \textbf{Input:const uint8\_t threads, const bool background}\\
\textbf{return: void}\\
Do the work of computing all different blocks in multithread. If \textit{background} is true, then return immediately, else return only when the job is done.\\
If \textit{threads} is $0$, try to guess the maximum threads from the number of available cpus.

\item [precompute\_max\_memory] \textbf{return: size\_t}\\
Each precomputation can take a lot of memory, depending on the configuration, so you might want to limit the number of precomputations run in parallel
depending on the memory used. This method returns the amount of memory taken by \textbf{ONE} precomputation.

\item [free] \textbf{Input: const uint8\_t sbn)}\\
\textbf{return: void}\\
Each block takes some memory, (a bit more than $symbols * symbol_size$), so once you are done sending source and repair symbols for one block,
you might want to free the memory of that block.

\item[blocks()] \textbf{return: uint8\_t} The number of blocks.

\item[block\_size()] \textbf{Input: const uint8\_t sbn}\\
\textbf{return: uint32\_t}\\
The block size, in bytes. Each block can have different symbols and thus different size.

\item[symbol\_size()] \textbf{return: uint16\_t} The size of a symbol.

\item[symbols] \textbf{Input:uint8\_t sbn}\\
\textbf{return: uint16\_t}\\
The number of symbols in a specific block. different blocks can have different symbols.

\item[max\_repair] \textbf{Input: const uint8\_t sbn)}\\
\textbf{return: uint32\_t}\\
The maximum amount of repair symbols that you can generate. Something less than $2^{24}$, but the exact number depends on the number of symbols
in a block
\end{description}
\newpage



\subsubsection{Blocks}
\index{Blocks!C++}
With the \textit{begin()/end()} calls you get \textit{Input iterators} to the blocks. a Block is has the following type:
\begin{verbatim}
template <typename Rnd_It, typename Out_It>
class Block
\end{verbatim}

and exposes the 4 following methods:
\begin{description}
\item[begin\_source]\textbf{return: Symbol\_Iterator}
\item[end\_source]\textbf{return: Symbol\_Iterator}
\item[begin\_repair]\textbf{return: Symbol\_Iterator}
\item[end\_repair]\textbf{Input: const uint32\_t max\_repair}\\
\textbf{return: Symbol\_Iterator}
\end{description}

As the names explain, you will get an iterator to the symbols in the block. As the number of repair symbols can vary, for now you get two separate begin/ends,
so that you can check when you sent the source symbols, and how many repair symbols you send.


\subsubsection{Symbols}
\index{Symbols!C++}
Finally, through the \textit{Symbol\_Iterator} \textit{Input Iterator} we get the \textbf{Symbol} class:
\begin{verbatim}
template <typename Rnd_It, typename Out_It>
class Symbol
\end{verbatim}

which exposes the 2 methods we need to get the symbol data:

\begin{description}
\item[operator*]\textbf{Input:Out\_It \&start, const Out\_It end}\\
\textbf{return: uint64\_t}\\
takes an output iterator, and fill it with the symbol data. returns the number of written iterators.
\item[id()]\textbf{return: uint32\_t}\\
return the id (\textit{$sbn + esi$}) of this symbol, that you need to include in every packet you send, before the symbols.
\end{description}

\subsubsection{The Decoder}
\index{Decoder!C++}

The decoder is a bit simpler than the encoder.

Theere are two constructors for the Decoder:

\begin{verbbox}[\small]
 std::vector<uint32_t> input, output;
 ...
 using T_it = typename std::vector<uint32_t>::iterator;
 RaptorQ::Decoder<T_it, T_it> dec (const OTI_Common_Data common,
                          const OTI_Scheme_Specific_Data scheme)
                              
 RaptorQ::Decoder<T_it, T_it> dec (uint64_t size,
              uint16_t symbol_size,  uint16_t sub_blocks,
                                                   uint8_t blocks)
\end{verbbox}
\theverbbox

Which should be pretty self-explanatory, once you understand how the encoder works.

The remaining methods are:
\begin{description}
\item[decode]\textbf{Input: Out\_It \&start, const Out\_It end}\\
\textbf{return:uint32\_t}\\
Write \textbf{all} the blocks into the iterator. refuses to write if the input has not been completely received. Return the number of iterators written.

\item[decode]\textbf{Input: Out\_It \&start, const Out\_It end, const uint8\_t sbn}\\
\textbf{return:uint32\_t}\\
Write a specific block into the iterator. Refuses to write if the input for that block has not been completely received.

\item[add\_symbol]\textbf{In\_It \&start, const In\_It end, const uint32\_t esi, const uint8\_t sbn}\\
\textbf{return: bool}\\
Add one symbol, while explicitly specifying the symbol id and the block id.

\item[add\_symbol]\textbf{In\_It \&start, const In\_It end, const uint32\_t id}\\
\textbf{return: bool}\\
Same as before, but extract the block id and the symbol id from the \textit{id} parameter

\item[free]\textbf{Input: const uint8\_t sbn}\\
\textbf{return: void}\\
You might have stopped using a block, but the memory is still there. free it.

\item[blocks()] \textbf{return: uint8\_t} The number of blocks.

\item[block\_size()] \textbf{Input: const uint8\_t sbn}\\
\textbf{return: uint32\_t}\\
The block size, in bytes. Each block can have different symbols and thus different size.

\item[symbol\_size()] \textbf{return: uint16\_t} The size of a symbol.

\item[symbols] \textbf{Input:uint8\_t sbn}\\
\textbf{return: uint16\_t}\\
The number of symbols in a specific block. different blocks can have different symbols.
\end{description}

\newpage
\subsection{C interface}
The C interface looks a lot like the C++ one.

First of all, you need to build the encoder or the decoder:

The C interface is just a wrapper around the C++ code, so you still have to specify the same things as before.
A quick glance at the constructors should give you all the information you need:

\marginlabel{C Constructors}
\begin{lstlisting}[language=C]
typedef enum { NONE = 0,
         ENC_8 = 1, ENC_16 = 2, ENC_32 = 3, ENC_64 = 4,
         DEC_8 = 5, DEC_16 = 6, DEC_32 = 7, DEC_64 = 8}
              RaptorQ_type;

struct RAPTORQ_LOCAL RaptorQ_ptr
{
  void *ptr = nullptr;
  const RaptorQ_type type;
};

RaptorQ_ptr* RaptorQ_Enc (const RaptorQ_type type,
                    void *data,
                    const uint64_t size,
                    const uint16_t min_subsymbol_size,
                    const uint16_t symbol_size,
                    const size_t max_memory);

RaptorQ_ptr* RaptorQ_Dec (const RaptorQ_type type,
              const RaptorQ_OTI_Common_Data common,
              const RaptorQ_OTI_Scheme_Specific_Data
                                                scheme);
\end{lstlisting}

The encoder and decoder must have a specific alignment. in C++ you can also have different
alignments for the input and output, while in C things are a bit more strict as we have to enumerate
all the possible cases. So you only get the same data alignment for both input and output.\\
Still, you don't lose anything in performance.\\

\newpage
\subsubsection{Common functions for (de/en)coding}

These functions are used by both the decoder and the decoder, and will be helpful in tracking how much memory
you will need to allocate, or in general in managing the encoder and decoder.

The \textbf{ptr} must be a valid encoder or decoder.
The names for now are self-explanatory. Blocks can have different symbols, so the size of a block and the number of symbols in a block
depend on which block we are talking about.

\marginlabel{symbols, blocks, memory}
\begin{lstlisting}[language=C]
uint16_t RaptorQ_symbol_size (struct RaptorQ_ptr *ptr);
uint8_t RaptorQ_blocks (struct RaptorQ_ptr *ptr);
uint32_t RaptorQ_block_size (struct RaptorQ_ptr *ptr,
                                    const uint8_t sbn);
uint16_t RaptorQ_symbols (struct RaptorQ_ptr *ptr,
                                    const uint8_t sbn);

void RaptorQ_free (struct RaptorQ_ptr **ptr);
void RaptorQ_free_block (struct RaptorQ_ptr *ptr,
                                    const uint8_t sbn);
\end{lstlisting}
Finally, when you are done working with a block, you can free the memory associated with the single block
and just free the whole (en/de)coder when you are done. Freeing the whole (en/de)coder will obviously free
also all the blocks.


\newpage
\subsubsection{Encoding}

\marginlabel{OTI Data}
First, we need to tell the receiver all the parameters that the encoder is using, and for that two functions are provided:

\begin{lstlisting}[language=C]
typedef uint64_t RaptorQ_OTI_Common_Data;
typedef uint32_t RaptorQ_OTI_Scheme_Specific_Data;

RaptorQ_OTI_Common_Data RaptorQ_OTI_Common (
                               struct RaptorQ_ptr *enc);
RaptorQ_OTI_Scheme_Specific_Data RaptorQ_OTI_Scheme (
                               struct RaptorQ_ptr *enc);
\end{lstlisting}

\index{Encoder!C}\marginlabel{Encoding}

\begin{lstlisting}[language=C]
// maximum number of repair symbol in a block
uint32_t RaptorQ_max_repair (RaptorQ_ptr *enc,
                                   const uint8_t sbn);
//estimate bytes of ram used in the precomputation
// of one block
size_t RaptorQ_precompute_max_memory (
                               struct RaptorQ_ptr *enc);
// do the precomputation.
void RaptorQ_precompute (struct RaptorQ_ptr *enc,
                                 const uint8_t threads,
                                 const bool background);

// encode one symbol. source block number and
// symbol id are in the ``id'' field.
// returns number of alignments written.
uint64_t RaptorQ_encode_id (struct RaptorQ_ptr *enc,
                                  void **data,
                                  const uint64_t size,
                                  const uint32_t id);
// encode one symbol. same as before
uint64_t RaptorQ_encode (struct RaptorQ_ptr *enc,
                                  void **data,
                                  const uint64_t size,
                                  const uint32_t esi,
                                  const uint8_t sbn);
// build an ``id'' field out of an esi and sbn field.
uint32_t RaptorQ_id (const uint32_t esi,
                                  const uint8_t sbn);
\end{lstlisting}
As for the C++ version, everything is thread-safe.

You can start the precomputation in background and not worry about it.\\
If you request repair symbols before the computation is finished, the call will block until the data is available.

The \textbf{encode} functions are the same, and will encode \textbf{one} symbol.
They work for both source symbols and repair symbols, just keep increasing the \textit{esi} field


\newpage
\subsubsection{Decoding}
\index{Decoder!C}\marginlabel{Decoding}
\begin{lstlisting}[language=C]
// decode all blocks.
// returns the number of written alignments.
// returns 0 if encoding of *everything* is not possible
uint64_t RaptorQ_decode (struct RaptorQ_ptr *dec,
                                   void **data,
                                   const size_t size);
// decode only one block (if possible)
// returns 0 if encoding of the block is not possible.
uint64_t RaptorQ_decode_sbn (struct RaptorQ_ptr *dec,
                                   void **data,
                                   const size_t size,
                                   const uint8_t sbn);

// add a received symbol the the structure.
// either by using the `id' field
bool RaptorQ_add_symbol_id (struct RaptorQ_ptr *dec,
                                   void **data,
                                   const uint32_t size,
                                   const uint32_t id);
// or by explicitely declaring esi and sbn
bool RaptorQ_add_symbol (struct RaptorQ_ptr *dec,
                                   void **data,
                                   const uint32_t size,
                                   const uint32_t esi,
                                   const uint8_t sbn);
\end{lstlisting}



%!TEX root = libRaptorQ.tex"`
\newpage
\fullpage
\section{GNU Free Documentation License}
\begin{verbatim}

                GNU Free Documentation License
                 Version 1.3, 3 November 2008


 Copyright (C) 2000, 2001, 2002, 2007, 2008 Free Software Foundation, Inc.
     <http://fsf.org/>
 Everyone is permitted to copy and distribute verbatim copies
 of this license document, but changing it is not allowed.

0. PREAMBLE

The purpose of this License is to make a manual, textbook, or other
functional and useful document "free" in the sense of freedom: to
assure everyone the effective freedom to copy and redistribute it,
with or without modifying it, either commercially or noncommercially.
Secondarily, this License preserves for the author and publisher a way
to get credit for their work, while not being considered responsible
for modifications made by others.

This License is a kind of "copyleft", which means that derivative
works of the document must themselves be free in the same sense.  It
complements the GNU General Public License, which is a copyleft
license designed for free software.

We have designed this License in order to use it for manuals for free
software, because free software needs free documentation: a free
program should come with manuals providing the same freedoms that the
software does.  But this License is not limited to software manuals;
it can be used for any textual work, regardless of subject matter or
whether it is published as a printed book.  We recommend this License
principally for works whose purpose is instruction or reference.


1. APPLICABILITY AND DEFINITIONS

This License applies to any manual or other work, in any medium, that
contains a notice placed by the copyright holder saying it can be
distributed under the terms of this License.  Such a notice grants a
world-wide, royalty-free license, unlimited in duration, to use that
work under the conditions stated herein.  The "Document", below,
refers to any such manual or work.  Any member of the public is a
licensee, and is addressed as "you".  You accept the license if you
copy, modify or distribute the work in a way requiring permission
under copyright law.

A "Modified Version" of the Document means any work containing the
Document or a portion of it, either copied verbatim, or with
modifications and/or translated into another language.

A "Secondary Section" is a named appendix or a front-matter section of
the Document that deals exclusively with the relationship of the
publishers or authors of the Document to the Document's overall
subject (or to related matters) and contains nothing that could fall
directly within that overall subject.  (Thus, if the Document is in
part a textbook of mathematics, a Secondary Section may not explain
any mathematics.)  The relationship could be a matter of historical
connection with the subject or with related matters, or of legal,
commercial, philosophical, ethical or political position regarding
them.

The "Invariant Sections" are certain Secondary Sections whose titles
are designated, as being those of Invariant Sections, in the notice
that says that the Document is released under this License.  If a
section does not fit the above definition of Secondary then it is not
allowed to be designated as Invariant.  The Document may contain zero
Invariant Sections.  If the Document does not identify any Invariant
Sections then there are none.

The "Cover Texts" are certain short passages of text that are listed,
as Front-Cover Texts or Back-Cover Texts, in the notice that says that
the Document is released under this License.  A Front-Cover Text may
be at most 5 words, and a Back-Cover Text may be at most 25 words.

A "Transparent" copy of the Document means a machine-readable copy,
represented in a format whose specification is available to the
general public, that is suitable for revising the document
straightforwardly with generic text editors or (for images composed of
pixels) generic paint programs or (for drawings) some widely available
drawing editor, and that is suitable for input to text formatters or
for automatic translation to a variety of formats suitable for input
to text formatters.  A copy made in an otherwise Transparent file
format whose markup, or absence of markup, has been arranged to thwart
or discourage subsequent modification by readers is not Transparent.
An image format is not Transparent if used for any substantial amount
of text.  A copy that is not "Transparent" is called "Opaque".

Examples of suitable formats for Transparent copies include plain
ASCII without markup, Texinfo input format, LaTeX input format, SGML
or XML using a publicly available DTD, and standard-conforming simple
HTML, PostScript or PDF designed for human modification.  Examples of
transparent image formats include PNG, XCF and JPG.  Opaque formats
include proprietary formats that can be read and edited only by
proprietary word processors, SGML or XML for which the DTD and/or
processing tools are not generally available, and the
machine-generated HTML, PostScript or PDF produced by some word
processors for output purposes only.

The "Title Page" means, for a printed book, the title page itself,
plus such following pages as are needed to hold, legibly, the material
this License requires to appear in the title page.  For works in
formats which do not have any title page as such, "Title Page" means
the text near the most prominent appearance of the work's title,
preceding the beginning of the body of the text.

The "publisher" means any person or entity that distributes copies of
the Document to the public.

A section "Entitled XYZ" means a named subunit of the Document whose
title either is precisely XYZ or contains XYZ in parentheses following
text that translates XYZ in another language.  (Here XYZ stands for a
specific section name mentioned below, such as "Acknowledgements",
"Dedications", "Endorsements", or "History".)  To "Preserve the Title"
of such a section when you modify the Document means that it remains a
section "Entitled XYZ" according to this definition.

The Document may include Warranty Disclaimers next to the notice which
states that this License applies to the Document.  These Warranty
Disclaimers are considered to be included by reference in this
License, but only as regards disclaiming warranties: any other
implication that these Warranty Disclaimers may have is void and has
no effect on the meaning of this License.

2. VERBATIM COPYING

You may copy and distribute the Document in any medium, either
commercially or noncommercially, provided that this License, the
copyright notices, and the license notice saying this License applies
to the Document are reproduced in all copies, and that you add no
other conditions whatsoever to those of this License.  You may not use
technical measures to obstruct or control the reading or further
copying of the copies you make or distribute.  However, you may accept
compensation in exchange for copies.  If you distribute a large enough
number of copies you must also follow the conditions in section 3.

You may also lend copies, under the same conditions stated above, and
you may publicly display copies.


3. COPYING IN QUANTITY

If you publish printed copies (or copies in media that commonly have
printed covers) of the Document, numbering more than 100, and the
Document's license notice requires Cover Texts, you must enclose the
copies in covers that carry, clearly and legibly, all these Cover
Texts: Front-Cover Texts on the front cover, and Back-Cover Texts on
the back cover.  Both covers must also clearly and legibly identify
you as the publisher of these copies.  The front cover must present
the full title with all words of the title equally prominent and
visible.  You may add other material on the covers in addition.
Copying with changes limited to the covers, as long as they preserve
the title of the Document and satisfy these conditions, can be treated
as verbatim copying in other respects.

If the required texts for either cover are too voluminous to fit
legibly, you should put the first ones listed (as many as fit
reasonably) on the actual cover, and continue the rest onto adjacent
pages.

If you publish or distribute Opaque copies of the Document numbering
more than 100, you must either include a machine-readable Transparent
copy along with each Opaque copy, or state in or with each Opaque copy
a computer-network location from which the general network-using
public has access to download using public-standard network protocols
a complete Transparent copy of the Document, free of added material.
If you use the latter option, you must take reasonably prudent steps,
when you begin distribution of Opaque copies in quantity, to ensure
that this Transparent copy will remain thus accessible at the stated
location until at least one year after the last time you distribute an
Opaque copy (directly or through your agents or retailers) of that
edition to the public.

It is requested, but not required, that you contact the authors of the
Document well before redistributing any large number of copies, to
give them a chance to provide you with an updated version of the
Document.


4. MODIFICATIONS

You may copy and distribute a Modified Version of the Document under
the conditions of sections 2 and 3 above, provided that you release
the Modified Version under precisely this License, with the Modified
Version filling the role of the Document, thus licensing distribution
and modification of the Modified Version to whoever possesses a copy
of it.  In addition, you must do these things in the Modified Version:

A. Use in the Title Page (and on the covers, if any) a title distinct
   from that of the Document, and from those of previous versions
   (which should, if there were any, be listed in the History section
   of the Document).  You may use the same title as a previous version
   if the original publisher of that version gives permission.
B. List on the Title Page, as authors, one or more persons or entities
   responsible for authorship of the modifications in the Modified
   Version, together with at least five of the principal authors of the
   Document (all of its principal authors, if it has fewer than five),
   unless they release you from this requirement.
C. State on the Title page the name of the publisher of the
   Modified Version, as the publisher.
D. Preserve all the copyright notices of the Document.
E. Add an appropriate copyright notice for your modifications
   adjacent to the other copyright notices.
F. Include, immediately after the copyright notices, a license notice
   giving the public permission to use the Modified Version under the
   terms of this License, in the form shown in the Addendum below.
G. Preserve in that license notice the full lists of Invariant Sections
   and required Cover Texts given in the Document's license notice.
H. Include an unaltered copy of this License.
I. Preserve the section Entitled "History", Preserve its Title, and add
   to it an item stating at least the title, year, new authors, and
   publisher of the Modified Version as given on the Title Page.  If
   there is no section Entitled "History" in the Document, create one
   stating the title, year, authors, and publisher of the Document as
   given on its Title Page, then add an item describing the Modified
   Version as stated in the previous sentence.
J. Preserve the network location, if any, given in the Document for
   public access to a Transparent copy of the Document, and likewise
   the network locations given in the Document for previous versions
   it was based on.  These may be placed in the "History" section.
   You may omit a network location for a work that was published at
   least four years before the Document itself, or if the original
   publisher of the version it refers to gives permission.
K. For any section Entitled "Acknowledgements" or "Dedications",
   Preserve the Title of the section, and preserve in the section all
   the substance and tone of each of the contributor acknowledgements
   and/or dedications given therein.
L. Preserve all the Invariant Sections of the Document,
   unaltered in their text and in their titles.  Section numbers
   or the equivalent are not considered part of the section titles.
M. Delete any section Entitled "Endorsements".  Such a section
   may not be included in the Modified Version.
N. Do not retitle any existing section to be Entitled "Endorsements"
   or to conflict in title with any Invariant Section.
O. Preserve any Warranty Disclaimers.

If the Modified Version includes new front-matter sections or
appendices that qualify as Secondary Sections and contain no material
copied from the Document, you may at your option designate some or all
of these sections as invariant.  To do this, add their titles to the
list of Invariant Sections in the Modified Version's license notice.
These titles must be distinct from any other section titles.

You may add a section Entitled "Endorsements", provided it contains
nothing but endorsements of your Modified Version by various
parties--for example, statements of peer review or that the text has
been approved by an organization as the authoritative definition of a
standard.

You may add a passage of up to five words as a Front-Cover Text, and a
passage of up to 25 words as a Back-Cover Text, to the end of the list
of Cover Texts in the Modified Version.  Only one passage of
Front-Cover Text and one of Back-Cover Text may be added by (or
through arrangements made by) any one entity.  If the Document already
includes a cover text for the same cover, previously added by you or
by arrangement made by the same entity you are acting on behalf of,
you may not add another; but you may replace the old one, on explicit
permission from the previous publisher that added the old one.

The author(s) and publisher(s) of the Document do not by this License
give permission to use their names for publicity for or to assert or
imply endorsement of any Modified Version.


5. COMBINING DOCUMENTS

You may combine the Document with other documents released under this
License, under the terms defined in section 4 above for modified
versions, provided that you include in the combination all of the
Invariant Sections of all of the original documents, unmodified, and
list them all as Invariant Sections of your combined work in its
license notice, and that you preserve all their Warranty Disclaimers.

The combined work need only contain one copy of this License, and
multiple identical Invariant Sections may be replaced with a single
copy.  If there are multiple Invariant Sections with the same name but
different contents, make the title of each such section unique by
adding at the end of it, in parentheses, the name of the original
author or publisher of that section if known, or else a unique number.
Make the same adjustment to the section titles in the list of
Invariant Sections in the license notice of the combined work.

In the combination, you must combine any sections Entitled "History"
in the various original documents, forming one section Entitled
"History"; likewise combine any sections Entitled "Acknowledgements",
and any sections Entitled "Dedications".  You must delete all sections
Entitled "Endorsements".


6. COLLECTIONS OF DOCUMENTS

You may make a collection consisting of the Document and other
documents released under this License, and replace the individual
copies of this License in the various documents with a single copy
that is included in the collection, provided that you follow the rules
of this License for verbatim copying of each of the documents in all
other respects.

You may extract a single document from such a collection, and
distribute it individually under this License, provided you insert a
copy of this License into the extracted document, and follow this
License in all other respects regarding verbatim copying of that
document.


7. AGGREGATION WITH INDEPENDENT WORKS

A compilation of the Document or its derivatives with other separate
and independent documents or works, in or on a volume of a storage or
distribution medium, is called an "aggregate" if the copyright
resulting from the compilation is not used to limit the legal rights
of the compilation's users beyond what the individual works permit.
When the Document is included in an aggregate, this License does not
apply to the other works in the aggregate which are not themselves
derivative works of the Document.

If the Cover Text requirement of section 3 is applicable to these
copies of the Document, then if the Document is less than one half of
the entire aggregate, the Document's Cover Texts may be placed on
covers that bracket the Document within the aggregate, or the
electronic equivalent of covers if the Document is in electronic form.
Otherwise they must appear on printed covers that bracket the whole
aggregate.


8. TRANSLATION

Translation is considered a kind of modification, so you may
distribute translations of the Document under the terms of section 4.
Replacing Invariant Sections with translations requires special
permission from their copyright holders, but you may include
translations of some or all Invariant Sections in addition to the
original versions of these Invariant Sections.  You may include a
translation of this License, and all the license notices in the
Document, and any Warranty Disclaimers, provided that you also include
the original English version of this License and the original versions
of those notices and disclaimers.  In case of a disagreement between
the translation and the original version of this License or a notice
or disclaimer, the original version will prevail.

If a section in the Document is Entitled "Acknowledgements",
"Dedications", or "History", the requirement (section 4) to Preserve
its Title (section 1) will typically require changing the actual
title.


9. TERMINATION

You may not copy, modify, sublicense, or distribute the Document
except as expressly provided under this License.  Any attempt
otherwise to copy, modify, sublicense, or distribute it is void, and
will automatically terminate your rights under this License.

However, if you cease all violation of this License, then your license
from a particular copyright holder is reinstated (a) provisionally,
unless and until the copyright holder explicitly and finally
terminates your license, and (b) permanently, if the copyright holder
fails to notify you of the violation by some reasonable means prior to
60 days after the cessation.

Moreover, your license from a particular copyright holder is
reinstated permanently if the copyright holder notifies you of the
violation by some reasonable means, this is the first time you have
received notice of violation of this License (for any work) from that
copyright holder, and you cure the violation prior to 30 days after
your receipt of the notice.

Termination of your rights under this section does not terminate the
licenses of parties who have received copies or rights from you under
this License.  If your rights have been terminated and not permanently
reinstated, receipt of a copy of some or all of the same material does
not give you any rights to use it.


10. FUTURE REVISIONS OF THIS LICENSE

The Free Software Foundation may publish new, revised versions of the
GNU Free Documentation License from time to time.  Such new versions
will be similar in spirit to the present version, but may differ in
detail to address new problems or concerns.  See
http://www.gnu.org/copyleft/.

Each version of the License is given a distinguishing version number.
If the Document specifies that a particular numbered version of this
License "or any later version" applies to it, you have the option of
following the terms and conditions either of that specified version or
of any later version that has been published (not as a draft) by the
Free Software Foundation.  If the Document does not specify a version
number of this License, you may choose any version ever published (not
as a draft) by the Free Software Foundation.  If the Document
specifies that a proxy can decide which future versions of this
License can be used, that proxy's public statement of acceptance of a
version permanently authorizes you to choose that version for the
Document.

11. RELICENSING

"Massive Multiauthor Collaboration Site" (or "MMC Site") means any
World Wide Web server that publishes copyrightable works and also
provides prominent facilities for anybody to edit those works.  A
public wiki that anybody can edit is an example of such a server.  A
"Massive Multiauthor Collaboration" (or "MMC") contained in the site
means any set of copyrightable works thus published on the MMC site.

"CC-BY-SA" means the Creative Commons Attribution-Share Alike 3.0 
license published by Creative Commons Corporation, a not-for-profit 
corporation with a principal place of business in San Francisco, 
California, as well as future copyleft versions of that license 
published by that same organization.

"Incorporate" means to publish or republish a Document, in whole or in 
part, as part of another Document.

An MMC is "eligible for relicensing" if it is licensed under this 
License, and if all works that were first published under this License 
somewhere other than this MMC, and subsequently incorporated in whole or 
in part into the MMC, (1) had no cover texts or invariant sections, and 
(2) were thus incorporated prior to November 1, 2008.

The operator of an MMC Site may republish an MMC contained in the site
under CC-BY-SA on the same site at any time before August 1, 2009,
provided the MMC is eligible for relicensing.


ADDENDUM: How to use this License for your documents

To use this License in a document you have written, include a copy of
the License in the document and put the following copyright and
license notices just after the title page:

    Copyright (c)  YEAR  YOUR NAME.
    Permission is granted to copy, distribute and/or modify this document
    under the terms of the GNU Free Documentation License, Version 1.3
    or any later version published by the Free Software Foundation;
    with no Invariant Sections, no Front-Cover Texts, and no Back-Cover Texts.
    A copy of the license is included in the section entitled "GNU
    Free Documentation License".

If you have Invariant Sections, Front-Cover Texts and Back-Cover Texts,
replace the "with...Texts." line with this:

    with the Invariant Sections being LIST THEIR TITLES, with the
    Front-Cover Texts being LIST, and with the Back-Cover Texts being LIST.

If you have Invariant Sections without Cover Texts, or some other
combination of the three, merge those two alternatives to suit the
situation.

If your document contains nontrivial examples of program code, we
recommend releasing these examples in parallel under your choice of
free software license, such as the GNU General Public License,
to permit their use in free software.

\end{verbatim}


\printindex
\end{document}

